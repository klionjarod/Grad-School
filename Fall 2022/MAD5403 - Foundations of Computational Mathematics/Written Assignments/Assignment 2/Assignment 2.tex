\documentclass[10pt]{article}
\usepackage[utf8]{inputenc}
\usepackage{amsmath, amssymb, amsthm} %gives us the character \varnothing, and then lets us use \begin{proof}
\usepackage{amssymb} 
\usepackage{enumitem}
\usepackage{siunitx}

\title{Homework 2}
\author{Jarod Klion}
\date{October 21, 2022}

\usepackage{fullpage}
\newcommand\norm[1]{\lVert#1\rVert}
\newcommand\abs[1]{\lvert#1\rvert}


\begin{document}
\maketitle

%\setlength{\fboxsep}{2pt}\fbox{\rule{4cm}{3cm}}

%%%%%%%%%%%%%%%%%%%%%%%%%%%%%%%%%%%%%%%%%%%%%%%%%%%%%%%%%%%%%%%%%%%%%%%%%%%%%%%%%%
%%%%%%%%%%%%%%%%%%%%%%%%%%%%%%%%%%%%%%%%%%%%%%%%%%%%%%%%%%%%%%%%%%%%%%%%%%%%%%%%%%

\begin{enumerate} 
\item Following is often referred to as the equivalence of the norms. For any square matrix $A \in \mathbb{R}^{n \times n}$, prove the following relations:
	\begin{enumerate}[label=(\alph*)]
		\item $\frac{1}{n}K_2(A) \leq K_1(A) \leq nK_2(A)$ \\ 
		\newline
		We will start by proving some equivalences for vector norms using the Cauchy-Schwarz inequality and then translating it to matrix norms and subsequently condition numbers:
		\begin{equation} \label{Cauchy}
			\norm{x}_2^2 = \sum_{i=1}^{n} x_{i}^2 \leq (\sum_{i=1}^{n}|x_i|)(\sum_{i=1}^{n}|x_i|) = \norm{x}_{1}^2
		\end{equation}
		Proving an equivalence between the one- and two-norm:
		\begin{align}
			\norm{x}_1 &=  \sum_{i=1}^{n} \abs{x_i} * 1 \nonumber \\
					&\leq \sqrt{\sum_{i=1}^{n} \abs{x_i}^2 * \sum_{i=1}^{n} 1^2} \nonumber \\ 
					&\leq \left(\sum_{i=1}^{n} \abs{x_i}^2\right)^{1/2} \left(\sum_{i=1}^{n} \abs{1}^2\right)^{1/2} = \sqrt{n}\norm{x}_2 \nonumber \\
			\label{vec2Norm}
			\Rightarrow \norm{x}_1 &\leq \sqrt{n} \norm{x}_2
		\end{align}
		Proving equivalence between the one- and $\infty$-norm:
		\begin{align}
			\norm{x}_1 = \sum_{i=1}^n \abs{x_i} \leq \sum_{i=1}^n &\abs{\max_{j \leq n} x_j} \leq n \max_{1\leq j \leq n} x_j = n \norm{x}_{\infty} \nonumber \\
			\label{vecInfNorm1}
			\Rightarrow \frac{1}{n} \norm{x}_1 &\leq \norm{x}_{\infty}
		\end{align}
		Proving equivalence between the two- and $\infty$-norm:
		\begin{align}
			\label{vecInfNorm2}
			\norm{x}_2 = \sum_{i=1}^n \abs{x_{i}}^2 &\geq \max_{i} \abs{x_i} = \norm{x}_\infty \nonumber \\
			\norm{x}_\infty &\leq \norm{x}_2 \\
			\norm{x}_2^2 = \sum_{i=1}^n \abs{x_{i}}^2 &\leq \mbox{n} \max_{i} \abs{x_i}^2 = n \norm{x}_\infty^2 \nonumber \\
			\label{vecInfNorm3}
			\norm{x}_2 &\leq \sqrt{n}\norm{x}_\infty
		\end{align}

		Putting together the norm equivalences we've shown so far, we can show the norm equivalence between $\norm{x}_1$ and $\norm{x}_{\infty}$ to be:
		\begin{gather}
			\frac{1}{n}\norm{x}_1 \leq \norm{x}_{\infty} \leq \norm{x}_2 \leq \norm{x}_1 \leq \sqrt{n}\norm{x}_2 \leq n\norm{x}_{1} \nonumber \\
			\label{vecInfEquiv}
			\frac{1}{n}\norm{x}_1 \leq \norm{x}_{\infty} \leq n \norm{x}_1
		\end{gather}	

		Now, some matrix norm equivalences using inequalities \ref{Cauchy} \& \ref{vec2Norm}:
		\begin{align}
		\norm{Ax}_2 \leq \norm{Ax}_1 &\leq \norm{A}_1\norm{x}_1 \leq \norm{A}_1 \sqrt{n} \norm{x}_2 \nonumber \\
		\label{matLHSNorm}
		\Rightarrow \frac{1}{\sqrt{n}}\norm{A}_2 &\leq \norm{A}_1 \\
		\norm{Ax}_1 \leq \sqrt{n}\norm{Ax}_2 &\leq \sqrt{n}\norm{A}_2\norm{x}_2 \leq \sqrt{n} \norm{A}_2 \norm{x}_1 \nonumber \\
		\label{matRHSNorm}
		\Rightarrow \norm{A}_1 &\leq \sqrt{n} \norm{A}_2
		\end{align}
		
		Since the condition number of a matrix is defined as $K(A) = \norm{A}\norm{A^{-1}}$, follow a similar process as above and multiply the norms to get the equivalence:
		\begin{equation}
			\frac{1}{n}K_2(A) \leq K_1(A) \leq nK_2(A)  \mbox{ QED}
		\end{equation}

		\item $\frac{1}{n^2}K_1(A) \leq K_{\inf}(A) \leq n^2K_1(A)$ \\
		\newline
		 
		Translating inequality \ref{vecInfEquiv} into matrix norms and the subsequent condition numbers, we can see that we arrive at the desired expression:
		\begin{equation}
			\frac{1}{n^2}K_1(A) \leq K_{\infty}(A) \leq n^2K_1(A)
		\end{equation}
	\end{enumerate}
\clearpage

%%%%%%%%%%%%%%%%%%%%%%%%%%%%%%%%%%%%%%%%%%%%%%%%%%%%%%%%%%%%%%%%%%%%%%%%%%%%%%%%%%
%%%%%%%%%%%%%%%%%%%%%%%%%%%%%%%%%%%%%%%%%%%%%%%%%%%%%%%%%%%%%%%%%%%%%%%%%%%%%%%%%%

\item Let $A$ be a sparse matrix of order $n$. Prove that the computational cost of the LU factorization of A is given by,
	\begin{equation*}
		\frac{1}{2} \sum_{k=1}^{n} l_k (A)(l_k(A) + 3) \mbox{ flops} ,
	\end{equation*}
	where $l_k(A)$ is the number of active rows at the $k$-th step of the factorization (i.e, the number of rows of $A$ with $i > k$ and $a_{ik} \neq 0$, and having accounted for all nonzero entries.
\end{enumerate}
\clearpage %Gives us a page break before the next section. Optional.
\end{document}