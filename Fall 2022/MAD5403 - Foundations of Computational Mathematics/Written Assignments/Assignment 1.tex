\documentclass{article}
\usepackage[utf8]{inputenc}
\usepackage{amsmath, amssymb, amsthm} %gives us the character \varnothing, and then lets us use \begin{proof}
\usepackage{amssymb} 
\usepackage{enumitem}

\title{Homework 1}
\author{Jarod Klion}
\date{September 22, 2022}
%This information doesn't actually show up on your document unless you use the maketitle command below

\usepackage{fullpage}

\begin{document}
\maketitle %This command prints the title based on information entered above

%Section and subsection automatically number unless you put the asterisk next to them.
%Basically, you type whatever text you want and use the $ sign to enter "math mode".
%For fancy calligraphy letters, use \mathbb{}
%Special characters are their own commands

\section*{Problem 1:}
Derive the relative and absolute condition number of a function at the point \emph{x}. Based on our notation from the class, note that the change of data $\delta d$ is equivalent to the change in the fucntion value, i.e $f(x)$. In particular, let's assume we perturb $x$ by $h > 0$, where $\delta d = f(x + h) - f(x)$.
\begin{enumerate}[label=(\alph*)]
	\item To show that $(\mathbb{R}\backslash\{-1\},*)$ is an Abelian group, we must prove it satisfies five properties:
		 
	\item $3 * x * x = 15$ \\
	         $3 * (x * x) = 3 * (x^2 + 2x) = 3(x^2+2x) + 3 + x^2 + 2x$ \\
	         $3x^2 + 6x + 3 + x^2 + 2x = 4x^2 + 8x + 3$ \\
	         $4x^2 + 8x +3 = 15 \rightarrow x^2 + 2x - 3 =  $ \\
	         $(x+3)(x-1) = 0$ \\
	         $ x=1,-3$
\end{enumerate}
\clearpage %Gives us a page break before the next section. Optional.
\end{document}